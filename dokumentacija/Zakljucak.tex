\begin{flushleft}
	
\end{flushleft}\chapter{Zaključak i budući rad}
		
		Cilj našeg projekta bio je razviti web stranicu koja omogućava lakšu koordinaciju i praćenje životinja u divljini. Aplikacija omogućuje različitim korisnicima da sudjeluju u akcijama, prate pozicije životinja te, u nekim ulogama, koordiniraju tragače u istraživačkim projektima.

Prva faza projekta bila je okupljanje tima te upoznavanje s dobivenim materijalima i zadatkom. Nakon prvog upoznavanja, počelo je dodjeljivanje uloga i rad na projektu. Kvalitetno odrađena organizacija uz pomoć koje je svaki član znao svoj posao značajno je olakšala i ubrzala rad na sadašnjem projektu. Izrađeni dijagrami i obrasci na početku rada bili su od velike pomoći pri pisanju koda i razvijanju backenda i frontenda.

Druga faza, iako kraća, bila je intenzivnija. Pri dodavanju različitih funkcionalnosti, trebali smo se samostalno upoznavati s odabranim alatima kako bismo ispunili ciljeve projekta. Uz to, dokumentirali smo ostale UML dijagrame i dovršili dokumentaciju koja će omogućiti budućim korisnicima lakše snalaženje.

Tijekom razvoja prepoznati su razni tehnički izazovi. Implementacija interaktivne karte koja prati lokaciju životinja i tragača predstavljala je kompleksan zadatak, zahtijevajući dobro poznavanje i integraciju s geolokacijskim rutama te vizualizacijom ruta. Još jedan izazov bio je sustav odobravanja registracija od strane administratora, što je zahtijevalo sigurnosne i administrativne funkcionalnosti.

Unatoč izazovima, uspješno je implementirana interaktivna karta koristeći se tehnologijama poput Reacta i Leafleta. Sustav odobravanja registracija također je uspješno implementiran, pružajući siguran pristup. Kroz projekt stečena su znanja o radu s geolokacijskim podacima, upravljanju korisnicima i implementaciji kompleksnih funkcionalnosti unutar web stranice.

Zadatak je proveden unutar planiranog vremenskog okvira, ostvarujući ključne funkcionalnosti. Tehnički izazovi bili su prepoznati i riješeni. Stečena su iskustva u radu s geolokacijskim tehnologijama i implementaciji složenih sustava.

Projekt je bio izazovan, ali i izrazito koristan za tim. Ovaj projekt je pružio priliku za razvoj vještina u području web aplikacija.
		
		
		
		\eject 