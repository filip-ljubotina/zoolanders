\chapter{Implementacija i korisničko sučelje}
		
		
		\section{Korištene tehnologije i alati}
		
			Tim je koristio \underline{WhatsApp}.\footnote[1]{https://www.whatsapp.com/} i \underline{Notion}.\footnote[2]{https://www.notion.so/} za komunikaciju unutar tima. Za izradu UML dijagrama korišten je \underline{Astah Professional}.\footnote[3]{https://astah.net/products/astah-professional/}, dok je za upravljanje izvornim kodom odabran \underline{Git}.\footnote[4]{https://git-scm.com/}. Udaljeni repozitorij projekta dostupan je na web platformi \underline{GitHub}.\footnote[5]{https://github.com/}.
			
			Kao razvojna okruženja korišteni su \underline{Visual Studio Code}.\footnote[6]{https://code.visualstudio.com/} i \underline{IntelliJ IDEA}.\footnote[7]{https://www.jetbrains.com/idea/}. Visual Studio Code je jednostavan uređivač koda s podrškom za razvojne operacije poput debugiranja, pokretanja zadataka i kontrolu verzija od tvrtke Microsoft, dok je IntelliJ popularno razvojno okruženje (IDE) razvijeno od strane tvrtke JetBrains. Namijenjeno je prvenstveno za rad s Java programskim jezikom.
			
			Aplikacija je napisana koristeći radni okvir \underline{Spring Boot}.\footnote[8]{https://spring.io/projects/spring-boot/} i jezik \underline{Java}.\footnote[9]{https://www.java.com/en/} za izradu backenda te \underline{React}.\footnote[10]{https://react.dev/} i jezik \underline{JavaScript}.\footnote[11]{https://www.javascript.com/} za izradu frontenda. React je besplatna i open-source JavaScript biblioteka za izgradnju korisničkih sučelja temeljenih na komponentama. Održava je Meta i zajednica pojedinačnih razvijatelja te tvrtki. Radni okvir Spring Boot je otvoreni, mikroservisni Java web okvir koji nudi Spring. Razvijen je kako bi pojednostavio proces izrade, konfiguracije i razmještanja Java aplikacija. 
			
			Baza podataka se nalazi na posluzitelju u oblaku \underline{Render}.\footnote[12]{https://render.com/}.
			
			
			\eject 
		
	
		\section{Ispitivanje programskog rješenja}
			
			\textbf{\textit{dio 2. revizije}}\\
			
			 \textit{U ovom poglavlju je potrebno opisati provedbu ispitivanja implementiranih funkcionalnosti na razini komponenti i na razini cijelog sustava s prikazom odabranih ispitnih slučajeva. Studenti trebaju ispitati temeljnu funkcionalnost i rubne uvjete.}
	
			
			\subsection{Ispitivanje komponenti}
			\textit{Potrebno je provesti ispitivanje jedinica (engl. unit testing) nad razredima koji implementiraju temeljne funkcionalnosti. Razraditi \textbf{minimalno 6 ispitnih slučajeva} u kojima će se ispitati redovni slučajevi, rubni uvjeti te izazivanje pogreške (engl. exception throwing). Poželjno je stvoriti i ispitni slučaj koji koristi funkcionalnosti koje nisu implementirane. Potrebno je priložiti izvorni kôd svih ispitnih slučajeva te prikaz rezultata izvođenja ispita u razvojnom okruženju (prolaz/pad ispita). }
			
			
			
			\subsection{Ispitivanje sustava}
			
			 \textit{Potrebno je provesti i opisati ispitivanje sustava koristeći radni okvir Selenium\footnote{\url{https://www.seleniumhq.org/}}. Razraditi \textbf{minimalno 4 ispitna slučaja} u kojima će se ispitati redovni slučajevi, rubni uvjeti te poziv funkcionalnosti koja nije implementirana/izaziva pogrešku kako bi se vidjelo na koji način sustav reagira kada nešto nije u potpunosti ostvareno. Ispitni slučaj se treba sastojati od ulaza (npr. korisničko ime i lozinka), očekivanog izlaza ili rezultata, koraka ispitivanja i dobivenog izlaza ili rezultata.\\ }
			 
			 \textit{Izradu ispitnih slučajeva pomoću radnog okvira Selenium moguće je provesti pomoću jednog od sljedeća dva alata:}
			 \begin{itemize}
			 	\item \textit{dodatak za preglednik \textbf{Selenium IDE} - snimanje korisnikovih akcija radi automatskog ponavljanja ispita	}
			 	\item \textit{\textbf{Selenium WebDriver} - podrška za pisanje ispita u jezicima Java, C\#, PHP koristeći posebno programsko sučelje.}
			 \end{itemize}
		 	\textit{Detalji o korištenju alata Selenium bit će prikazani na posebnom predavanju tijekom semestra.}
			
			\eject 
		
		
		\section{Dijagram razmještaja}
			
			\textbf{\textit{dio 2. revizije}}
			
			 \textit{Potrebno je umetnuti \textbf{specifikacijski} dijagram razmještaja i opisati ga. Moguće je umjesto specifikacijskog dijagrama razmještaja umetnuti dijagram razmještaja instanci, pod uvjetom da taj dijagram bolje opisuje neki važniji dio sustava.}
			
			\eject 
		
		\section{Upute za puštanje u pogon}
		
			\textbf{\textit{dio 2. revizije}}\\
		
			 \textit{U ovom poglavlju potrebno je dati upute za puštanje u pogon (engl. deployment) ostvarene aplikacije. Na primjer, za web aplikacije, opisati postupak kojim se od izvornog kôda dolazi do potpuno postavljene baze podataka i poslužitelja koji odgovara na upite korisnika. Za mobilnu aplikaciju, postupak kojim se aplikacija izgradi, te postavi na neku od trgovina. Za stolnu (engl. desktop) aplikaciju, postupak kojim se aplikacija instalira na računalo. Ukoliko mobilne i stolne aplikacije komuniciraju s poslužiteljem i/ili bazom podataka, opisati i postupak njihovog postavljanja. Pri izradi uputa preporučuje se \textbf{naglasiti korake instalacije uporabom natuknica} te koristiti što je više moguće \textbf{slike ekrana} (engl. screenshots) kako bi upute bile jasne i jednostavne za slijediti.}
			
			
			 \textit{Dovršenu aplikaciju potrebno je pokrenuti na javno dostupnom poslužitelju. Studentima se preporuča korištenje neke od sljedećih besplatnih usluga: \href{https://aws.amazon.com/}{Amazon AWS}, \href{https://azure.microsoft.com/en-us/}{Microsoft Azure} ili \href{https://www.heroku.com/}{Heroku}. Mobilne aplikacije trebaju biti objavljene na F-Droid, Google Play ili Amazon App trgovini.}
			
			
			\eject 