\chapter{Arhitektura i dizajn sustava}
		
		\textbf{\textit{dio 1. revizije}}\\

		\textit{ Potrebno je opisati stil arhitekture te identificirati: podsustave, preslikavanje na radnu platformu, spremišta podataka, mrežne protokole, globalni upravljački tok i sklopovsko-programske zahtjeve. Po točkama razraditi i popratiti odgovarajućim skicama:}
	\begin{itemize}
		\item 	\textit{izbor arhitekture temeljem principa oblikovanja pokazanih na predavanjima (objasniti zašto ste baš odabrali takvu arhitekturu)}
		\item 	\textit{organizaciju sustava s najviše razine apstrakcije (npr. klijent-poslužitelj, baza podataka, datotečni sustav, grafičko sučelje)}
		\item 	\textit{organizaciju aplikacije (npr. slojevi frontend i backend, MVC arhitektura) }		
	\end{itemize}

	Arhitektura se moze podijeliti na tri podsustava:
	
	\begin{itemize}
		\item Web posluzitelj
		\item Web aplikacija
		\item Baza podataka
	\end{itemize}

	
	\textit{Web preglednik} je alat koji omogućava korisnicima pregledavanje web stranica i njihovih povezanih multimedijalnih sadržaja. Svaki internetski preglednik djeluje kao prevoditelj, jer interpretira web stranice napisane u kodu kako bi ih prikazao korisnicima na razumljiv način. Korisnici putem web preglednika šalju zahtjeve web poslužitelju.
	
	\textit{Web poslužitelj} je ključni element u radu web aplikacije. Njegova glavna uloga je olakšavanje komunikacije između klijenta i aplikacije putem HTTP protokola. Poslužitelj pokreće web aplikaciju i prenosi joj zahtjev.
	
	\textit{Web aplikacija} služi korisniku za obradu željenih zahtjeva. Aplikacija obrađuje zahtjev, pristupa bazi podataka prema potrebi i putem poslužitelja vraća korisniku odgovor u obliku HTML dokumenta koji se prikazuje u web pregledniku.
	
	\textit{Baza podataka} ima svrhu pohranjivanja i upravljanja strukturiranim podacima koji se koriste u aplikaciji. Svaki put kad korisnik šalje zahtjev putem web preglednika, web aplikacija može pristupiti bazi podataka kako bi dohvatila ili ažurirala potrebne informacije. Baza podataka omogućava učinkovit pristup, pretraživanje i manipulaciju podacima, što je ključno za pravilan rad web aplikacije.
	
	Za izradu naše web aplikacije odabrali smo programski jezik Java zajedno s Springboot radnim okvirom, kao i programski jezik JavaScript. Razvojna okruženja koja koristimo su Visual Studio Code i IntelliJ. Arhitektura sustava temelji se na MVC (Model-View-Controller) konceptu, koji je podržan od strane Springboot radnog okvira i nudi gotove predloške koji olakšavaju razvoj web aplikacije. Kao poslužitelja baze podataka smo koristili PostgreSQL.
	
	MVC koncept donosi neovisnost u razvoju pojedinih dijelova aplikacije, što olakšava testiranje, kao i dodavanje novih svojstava u sustav. Sastoji se od:
	
	\begin{itemize}
		\item \textbf{Model} - Središnja komponenta sustava koja predstavlja dinamičke strukture podataka neovisne o korisničkom sučelju. Upravlja podacima, logikom i pravilima aplikacije, te prima ulazne podatke od Controllera.
		\item \textbf{View} - Ovdje se prikazuju podaci, primjerice u obliku grafova. Moguća su različita sučelja za prikaz informacija, poput grafičkog ili tabličnog prikaza podataka.
		\item \textbf{Controller} - Prima ulaze i prilagođava ih za daljnju interakciju s Modelom ili Viewom. Upravlja korisničkim zahtjevima i temeljem njih izvodi daljnje interakcije s ostalim elementima sustava.
	\end{itemize}
		

		

				
		\section{Baza podataka}
			
			\textbf{\textit{dio 1. revizije}}\\
			
		\textit{Potrebno je opisati koju vrstu i implementaciju baze podataka ste odabrali, glavne komponente od kojih se sastoji i slično.}
		
			\subsection{Opis tablica}
			

				\textit{Svaku tablicu je potrebno opisati po zadanom predlošku. Lijevo se nalazi točno ime varijable u bazi podataka, u sredini se nalazi tip podataka, a desno se nalazi opis varijable. Svjetlozelenom bojom označite primarni ključ. Svjetlo plavom označite strani ključ}
				
				
				\begin{longtblr}[
					label=none,
					entry=none
					]{
						width = \textwidth,
						colspec={|X[6,l]|X[6, l]|X[20, l]|}, 
						rowhead = 1,
					} %definicija širine tablice, širine stupaca, poravnanje i broja redaka naslova tablice
					\hline \SetCell[c=3]{c}{\textbf{korisnik - ime tablice}}	 \\ \hline[3pt]
					\SetCell{LightGreen}IDKorisnik & INT	&  	Lorem ipsum dolor sit amet, consectetur adipiscing elit, sed do eiusmod  	\\ \hline
					korisnickoIme	& VARCHAR &   	\\ \hline 
					email & VARCHAR &   \\ \hline 
					ime & VARCHAR	&  		\\ \hline 
					\SetCell{LightBlue} primjer	& VARCHAR &   	\\ \hline 
				\end{longtblr}
				
				
			
			\subsection{Dijagram baze podataka}
				\textit{ U ovom potpoglavlju potrebno je umetnuti dijagram baze podataka. Primarni i strani ključevi moraju biti označeni, a tablice povezane. Bazu podataka je potrebno normalizirati. Podsjetite se kolegija "Baze podataka".}
			
			\eject
			
			
		\section{Dijagram razreda}
		
			\textit{Potrebno je priložiti dijagram razreda s pripadajućim opisom. Zbog preglednosti je moguće dijagram razlomiti na više njih, ali moraju biti grupirani prema sličnim razinama apstrakcije i srodnim funkcionalnostima.}\\
			
			\textbf{\textit{dio 1. revizije}}\\
			
			\textit{Prilikom prve predaje projekta, potrebno je priložiti potpuno razrađen dijagram razreda vezan uz \textbf{generičku funkcionalnost} sustava. Ostale funkcionalnosti trebaju biti idejno razrađene u dijagramu sa sljedećim komponentama: nazivi razreda, nazivi metoda i vrste pristupa metodama (npr. javni, zaštićeni), nazivi atributa razreda, veze i odnosi između razreda.}\\
			
			\textbf{\textit{dio 2. revizije}}\\			
			
			\textit{Prilikom druge predaje projekta dijagram razreda i opisi moraju odgovarati stvarnom stanju implementacije}
			
			
			
			\eject
		
		\section{Dijagram stanja}
			
			
			\textbf{\textit{dio 2. revizije}}\\
			
			\textit{Potrebno je priložiti dijagram stanja i opisati ga. Dovoljan je jedan dijagram stanja koji prikazuje \textbf{značajan dio funkcionalnosti} sustava. Na primjer, stanja korisničkog sučelja i tijek korištenja neke ključne funkcionalnosti jesu značajan dio sustava, a registracija i prijava nisu. }
			
			
			\eject 
		
		\section{Dijagram aktivnosti}
			
			\textbf{\textit{dio 2. revizije}}\\
			
			 \textit{Potrebno je priložiti dijagram aktivnosti s pripadajućim opisom. Dijagram aktivnosti treba prikazivati značajan dio sustava.}
			
			\eject
		\section{Dijagram komponenti}
		
			\textbf{\textit{dio 2. revizije}}\\
		
			 \textit{Potrebno je priložiti dijagram komponenti s pripadajućim opisom. Dijagram komponenti treba prikazivati strukturu cijele aplikacije.}