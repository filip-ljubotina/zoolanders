\chapter{Arhitektura i dizajn sustava}
		
	Arhitektura se moze podijeliti na tri podsustava:
	
	\begin{itemize}
		\item Web posluzitelj
		\item Web aplikacija
		\item Baza podataka
	\end{itemize}

	
	\textit{Web preglednik} je alat koji omogućava korisnicima pregledavanje web stranica i njihovih povezanih multimedijalnih sadržaja. Svaki internetski preglednik djeluje kao prevoditelj, jer interpretira web stranice napisane u kodu kako bi ih prikazao korisnicima na razumljiv način. Korisnici putem web preglednika šalju zahtjeve web poslužitelju.
	
	\textit{Web poslužitelj} je ključni element u radu web aplikacije. Njegova glavna uloga je olakšavanje komunikacije između klijenta i aplikacije putem HTTP protokola. Poslužitelj pokreće web aplikaciju i prenosi joj zahtjev.
	
	\textit{Web aplikacija} služi korisniku za obradu željenih zahtjeva. Aplikacija obrađuje zahtjev, pristupa bazi podataka prema potrebi i putem poslužitelja vraća korisniku odgovor u obliku HTML dokumenta koji se prikazuje u web pregledniku.
	
	\textit{Baza podataka} ima svrhu pohranjivanja i upravljanja strukturiranim podacima koji se koriste u aplikaciji. Svaki put kad korisnik šalje zahtjev putem web preglednika, web aplikacija može pristupiti bazi podataka kako bi dohvatila ili ažurirala potrebne informacije. Baza podataka omogućava učinkovit pristup, pretraživanje i manipulaciju podacima, što je ključno za pravilan rad web aplikacije.
	
	Za izradu naše web aplikacije odabrali smo programski jezik Java zajedno s Springboot radnim okvirom, kao i programski jezik JavaScript. Razvojna okruženja koja koristimo su Visual Studio Code i IntelliJ. Arhitektura sustava temelji se na MVC (Model-View-Controller) konceptu, koji je podržan od strane Springboot radnog okvira i nudi gotove predloške koji olakšavaju razvoj web aplikacije. Kao poslužitelja baze podataka smo koristili PostgreSQL.
	
	MVC koncept donosi neovisnost u razvoju pojedinih dijelova aplikacije, što olakšava testiranje, kao i dodavanje novih svojstava u sustav. Sastoji se od:
	
	\begin{itemize}
		\item \textbf{Model} - Središnja komponenta sustava koja predstavlja dinamičke strukture podataka neovisne o korisničkom sučelju. Upravlja podacima, logikom i pravilima aplikacije, te prima ulazne podatke od Controllera.
		\item \textbf{View} - Ovdje se prikazuju podaci, primjerice u obliku grafova. Moguća su različita sučelja za prikaz informacija, poput grafičkog ili tabličnog prikaza podataka.
		\item \textbf{Controller} - Prima ulaze i prilagođava ih za daljnju interakciju s Modelom ili Viewom. Upravlja korisničkim zahtjevima i temeljem njih izvodi daljnje interakcije s ostalim elementima sustava.
	\end{itemize}
		

		

				
		\section{Baza podataka}
				
				Za bazu podataka koristi se relacijska baza podataka. Svaka tablica ima svoje ime i atribute. Vrste atributa koji se mogu nalaziti u tablici su primarni ključ, strani ključ ili atribut s nekom informacijom vezanom za tablicu. Baza podataka sastoji se od tablica:
		
				\begin{packed_item}
					\item AppUser
					\item ConfirmationToken
					\item StationManager
					\item SearcherInTheField
					\item Action
					\item Request
					\item Station
					\item Qualification
					\item Animal
					\item Location
				\end{packed_item}
				
				
		
			\subsection{Opis tablica}
			

				U tablici \textbf{AppUser} pohranjuju se podaci o svim korisnicima: \textit{id, userName, image, firstName, lastName, email, password, locked, enabled}. Primarni ključ je \textit{id}, i nema stranih ključeva. U njoj se nalaze svi korisnici, odnosno sve vrste korisnika. Korisnike dalje razlikujemo na tragače koji imaju svoje dodatne atribute u tablici \textbf{SearcherInTheField} i voditelje postaja koji imaju svoje dodatne atribute u tablici \textbf{StationManager}. Istraživači nemaju svoju posebnu tablicu jer je nepotrebna, nemaju svoje dodatne atribute koje moramo spremiti, a jedino za što su nam potrebni su akcije, a za to imamo id spremljen pod \textit{begun} u tablici \textbf{Action} pa tako dobijemo sve ostale podatke o istraživaču. S atributom \textit{id} je u odnosu One-to-One s tablicama \textbf{StationManager} i \textbf{SearcherInTheField} i u odnosu One-to-Many s tablicama \textbf{ConfirmationToken} i \textbf{Action}.

				
				
				\begin{longtblr}[
					label=none,
					entry=none
					]{
						width = \textwidth,
						colspec={|X[6,l]|X[6, l]|X[20, l]|}, 
						rowhead = 1,
					} %definicija širine tablice, širine stupaca, poravnanje i broja redaka naslova tablice
					\hline \SetCell[c=3]{c}{\textbf{AppUser}}	 \\ \hline[3pt]
					\SetCell{LightGreen}id & BIGINT	&  	id korisnika 	\\ \hline
					userName	& VARCHAR &  korisničko ime 	\\ \hline 
					image & BYTEA &  heksadekatski zapis korisničke slike  \\ \hline 
					firstName & VARCHAR	&  ime korisnika  \\ \hline 
					lastName & VARCHAR	&  prezime korisnika  \\ \hline 
					email & VARCHAR	&  email korisnika  \\ \hline 
					password & VARCHAR	&  lozinka korisnika  \\ \hline 
					locked & BOOLEAN & je li korisnika potvrdio admin \\ \hline
					enabled & BOOLEAN & je li korisnik potvrđen emailom \\ \hline
				\end{longtblr}
				
				U tablici \textbf{ConfirmationToken} pohranjuju se podaci za token za potvrdu poslanu korisniku: \textit{id, token, createdAt, expiresAt, confirmedAt, user}. Svaki token je povezan s korisnikom kojem je poslan preko \textit{user} u koji se sprema id korisnika. Primarni ključ je \textit{id}, a strani ključ je \textit{user}. S atributom \textit{user} je u odnosu Many-to-One s tablicom \textbf{AppUser}.
				
				\begin{longtblr}[
					label=none,
					entry=none
					]{
						width = \textwidth,
						colspec={|X[6,l]|X[6, l]|X[20, l]|}, 
						rowhead = 1,
					} %definicija širine tablice, širine stupaca, poravnanje i broja redaka naslova tablice
					\hline \SetCell[c=3]{c}{\textbf{ConfirmationToken}}	 \\ \hline[3pt]
					\SetCell{LightGreen}id & BIGINT	&  	id tokena 	\\ \hline
					token & VARCHAR & token \\ \hline
					createdAt & TIMESTAMP & vrijeme kada je token stvoren \\ \hline
					expiersAt & TIMESTAMP & vrijeme kada token ističe \\ \hline
					confirmedAt & TIMESTAMP & vrijeme kada je token potvrđen \\ \hline
					\SetCell{LightBlue}user	& BIGINT &  id korisnika kojemu je poslan token \\ \hline  
				\end{longtblr}

				U tablici \textbf{StationManager} pohranjuju se dodatni podaci za voditelja postaje: \textit{id, station}. U \textit{id} se sprema id korisnika koji je voditelj postaje, a u \textit{station} se sprema id postaje. Primarni ključ i ujedno i strani ključ je \textit{id} i također je strani ključ \textit{station}. S atributom \textit{id} je u odnosu One-to-One s tablicom \textbf{AppUser}, s atributom \textit{station} je u odnosu One-to-One s tablicom \textbf{Station}.

				
				\begin{longtblr}[
					label=none,
					entry=none
					]{
						width = \textwidth,
						colspec={|X[6,l]|X[6, l]|X[20, l]|}, 
						rowhead = 1,
					} %definicija širine tablice, širine stupaca, poravnanje i broja redaka naslova tablice

					\hline \SetCell[c=3]{c}{\textbf{StationManager}}	 \\ \hline[3pt]
					\SetCell{LightGreen}id & BIGINT	&  	id korisničkog računa voditelja 	\\ \hline
					\SetCell{LightBlue}station & BIGINT	&  	id postaje koju vodi voditelj 	\\ \hline
				\end{longtblr}
			

			U tablici \textbf{SearcherInTheField} pohranjuju se dodatni podaci za tragača: \textit{id, station, qualification}. U \textit{id} je spremljen id korisnika koji je tragač, u \textit{station} je pohranjen id postaje kojoj tragač pripada, a u \textit{qualification} je pohranjen id kvalifikacije koju tragač ima. Primarni ključ i ujedno i strani ključ je \textit{id} i također je strani ključ \textit{station} i \textit{qualification}. S atributom \textit{id} je u odnosu One-to-One s tablicom \textbf{AppUser}, s atributom \textit{station} je u odnosu Many-to-One s tablicom \textbf{Station}, s atributom \textit{qualification} je u odnosu Many-to-One s tablicom \textbf{Qualification}, s atributom \textit{id} je u odnosu One-to-Many s tablicom \textbf{Request} i  s atributom \textit{id} je u odnosu One-to-Many s tablicom \textbf{Location}.


			
				\begin{longtblr}[
					label=none,
					entry=none
					]{
						width = \textwidth,
						colspec={|X[10,l]|X[6, l]|X[20, l]|}, 
						rowhead = 1,
					} %definicija širine tablice, širine stupaca, poravnanje i broja redaka naslova tablice

					\hline \SetCell[c=3]{c}{\textbf{SearcherInTheField}}	 \\ \hline[3pt]
					\SetCell{LightGreen}id & BIGINT	&  	id korisničkog računa tragača 	\\ \hline
					\SetCell{LightBlue}station & BIGINT	&  	id postaje kojoj pripada 	\\ \hline
					\SetCell{LightBlue}qualification	& BIGINT &  id osposobljenosti tragača 	\\ \hline  
				\end{longtblr}
			

			U tablici \textbf{Action} pohranjuju se podaci o akciji: \textit{id, begun, description, comment}. U \textit{id} se sprema id akcije, u \textit{begun} se sprema id istraživača koji je pokrenuo akciju, odnosno id korisnika koji je pokrenuo akciju. \textit{Description} je opis akcije, odnosno što se radi i koji je cilj i \textit{comment} je dodatni komentar za akciju. Primarni ključ je \textit{id}, strani ključ je \textit{begun}. S atributom \textit{begun} je u odnosu Many-to-One s tablicom \textbf{AppUser} i s atributom \textit{id} je u odnosu One-to-Many s tablicom \textbf{Request}.


			
			\begin{longtblr}[
				label=none,
				entry=none
				]{
					width = \textwidth,
					colspec={|X[6,l]|X[6, l]|X[20, l]|}, 
					rowhead = 1,
				} %definicija širine tablice, širine stupaca, poravnanje i broja redaka naslova tablice
				
				\hline \SetCell[c=3]{c}{\textbf{Action}}	 \\ \hline[3pt]
				\SetCell{LightGreen}id & BIGINT	&  	id akcije 	\\ \hline
				\SetCell{LightBlue}begun & BIGINT	&  	id korisnika (istraživača) koji je započeo akciju 	\\ \hline
				description	& TEXT &  opis akcije 	\\ \hline
				comment & TEXT &  komentar za akciju 	\\ \hline  
			\end{longtblr}
			

			U tablici \textbf{Request} pohranjuju se podaci za zahtjev za tragačem: \textit{id, action, searcher, description, qualification, completed}. U \textit{id} je spremljen id zahtjeva, u \textit{action} je spremljen id akcije kojoj taj zahtjev pripada, odnosno zahtjevi se odnose na pojedini posao koji treba napraviti u pojedinoj akciji pa se u \textit{action} sprema id akcije kojoj taj posao pripada, \textit{description} služi kao opis posla koji tragač treba napraviti, u \textit{searcher} se sprema id tragača kojemu je dodijeljen posao, \textit{qualification} je id kvalifikacije potrebne za obavljanje posla i u \textit{completed} se sprema je li tragač izvršio taj posao ili nije. Primarni ključ je \textit{id}, strani ključevi su \textit{action}, \textit{searcher} i \textit{qualification}. S atributom \textit{action} je u odnosu Many-to-One s tablicom \textbf{Action}, s atributom \textit{searcher} je u odnosu Many-to-One s tablicom \textbf{SearcherInTheField} i s atributom \textit{qualification} je u odnosu One-to-One s tablicom \textbf{Qualification}.


			
			\begin{longtblr}[
				label=none,
				entry=none
				]{
					width = \textwidth,
					colspec={|X[6,l]|X[6, l]|X[20, l]|}, 
					rowhead = 1,
				} %definicija širine tablice, širine stupaca, poravnanje i broja redaka naslova tablice

				\hline \SetCell[c=3]{c}{\textbf{Request}}	 \\ \hline[3pt]
				\SetCell{LightGreen}id & BIGINT	&  	id zahtjeva 	\\ \hline
				\SetCell{LightBlue}action & BIGINT	&  	id akcije kojoj pripada zahtjev 	\\ \hline
				\SetCell{LightBlue}searcher & BIGINT	&  	id tragača koji je dobio zahtjev 	\\ \hline
				description	& TEXT &  opis  što tragač mora napraviti 	\\ \hline 
				\SetCell{LightBlue} qualification & TEXT & potrebne kvalifikacije \\ \hline
				completed & BOOLEAN & je li zahtjev završen ili ne \\ \hline
			\end{longtblr}
			

			U tablici \textbf{Station} pohranjuju se podaci za postaju: \textit{id, name}. U \textit{id} je spremljen id postaje, a u \textit{name} je spremljeno ime postaje. Primarni ključ je \textit{id} i nema stranih ključeva. S atributom \textit{id} je u odnosu One-to-One s tablicom \textbf{StationManager}, s atributom \textit{id} je u odnosu One-to-Many s tablicom \textbf{SearcherInTheField} i s atributom \textit{id} je u odnosu One-to-One s tablicom \textbf{Location}.


			
			\begin{longtblr}[
				label=none,
				entry=none
				]{
					width = \textwidth,
					colspec={|X[6,l]|X[6, l]|X[20, l]|}, 
					rowhead = 1,
				} %definicija širine tablice, širine stupaca, poravnanje i broja redaka naslova tablice

				\hline \SetCell[c=3]{c}{\textbf{Station}}	 \\ \hline[3pt]
				\SetCell{LightGreen}id & BIGINT	&  	id postaje 	\\ \hline
				name & VARCHAR & ime postaje \\ \hline
			\end{longtblr}

			U tablici \textbf{Qualification} pohranjuju se podaci za kvalifikacije: \textit{id, description, coverage, visibility}. U \textit{id} pohranjuje se id od kvalifikacije, u \textit{description} pohranjuje se opis kvalifikacije, u \textit{coverage} pohranjuje se područje pokriveno u kilometrima, za helikopter bi bilo npr. 10.00 km, u \textit{visibility} se sprema ocjena vidljivosti koja je obrnuto proporcionalna pokrivenosti, npr. helikopter ima veliku pokrivenost, ali se teško vide tragovi životinja pa ima manju vidljivost, dok hodanje šumom ima malu pokrivenost, ali se dobro vide tragovi koje ostave životinjei pa ima veliku vidljivost. Primarni ključ je \textit{id} i nema stranih ključeva. S atributom \textit{id} je u odnosu One-to-Many s tablicom \textbf{SearcherInTheField} i u odnosu One-to-One s tablicom \textbf{Request}.


			
			\begin{longtblr}[
				label=none,
				entry=none
				]{
					width = \textwidth,
					colspec={|X[10,l]|X[6, l]|X[20, l]|}, 
					rowhead = 1,
				} %definicija širine tablice, širine stupaca, poravnanje i broja redaka naslova tablice

				\hline \SetCell[c=3]{c}{\textbf{Qualification}}	 \\ \hline[3pt]
				\SetCell{LightGreen}id & BIGINT	&  	id osposobljenost 	\\ \hline
				description & VARCHAR & opis osposobljenosti (npr. pješice, autom,...) \\ \hline
				coverage & DOUBLE PRECISION & područje pokriveno u kilometrima \\ \hline
				visibility & BIGINT & ocjena vidljivosti na skali od 1 do 10 (1 je najgore, a 10 najbolje) \\ \hline
			\end{longtblr}
			

			U tablici \textbf{Animal} pohranjuju se podaci o životinji: \textit{id, species, comment}. U \textit{id} se pohranjuje id od životinje, u \textit{species} se pohranjuje vrsta životinje i u \textit{comment} se pohranjuje komentar o životinji, komentar može biti bilo što: izgled, ponašanje, zdravstveno stanje, ime životinje ili nešto drugo. Primarni ključ je \textit{id} i nema stranih ključeva. S atributom \textit{id} je u odnosu One-to-Many s tablicom \textbf{Location}.


			
			\begin{longtblr}[
				label=none,
				entry=none
				]{
					width = \textwidth,
					colspec={|X[6,l]|X[6, l]|X[20, l]|}, 
					rowhead = 1,
				} %definicija širine tablice, širine stupaca, poravnanje i broja redaka naslova tablice

				\hline \SetCell[c=3]{c}{\textbf{Animal}}	 \\ \hline[3pt]
				\SetCell{LightGreen}id & BIGINT	&  	id životinje 	\\ \hline
				species & VARCHAR & vrsta životinje \\ \hline
				comment & TEXT & komentar za životinju \\ \hline
			\end{longtblr}
			

			U tablici \textbf{Location} pohranjuju se podaci o lokaciji koja ima neku važnost: \textit{id, object, longitude, latitude, time, comment}. U \textit{id} se pohranjuje id lokacije, u \textit{object} se pohranjuje id objekta na nekoj lokaciji, objekt može biti tragač, životinja ili postaja. U \textit{longitude} i \textit{latitude} se spremaju geografska dužina i širina, u \textit{time} se sprema vrijeme kada je lokacija unesena i u \textit{comment} se sprema komentar za pojedinu lokaciju. Primarni ključ je \textit{id}, a strani ključ je \textit{object}. S atributom \textit{object} je u odnosu One-to-One s tablicom \textbf{Station}, Many-to-One s tablicom \textbf{Animal} i Many-to-One s tablicom \textbf{SearcherInTheField}.


			
			\begin{longtblr}[
				label=none,
				entry=none
				]{
					width = \textwidth,
					colspec={|X[6,l]|X[6, l]|X[20, l]|}, 
					rowhead = 1,
				} %definicija širine tablice, širine stupaca, poravnanje i broja redaka naslova tablice

				\hline \SetCell[c=3]{c}{\textbf{Location}}	 \\ \hline[3pt]
				\SetCell{LightGreen}id & BIGINT & id lokacije \\ \hline
				\SetCell{LightBlue}object & BIGINT	&  	id objekta na određenoj lokaciji 	\\ \hline
				longitude & DOUBLE PRECISION & geografska dužina \\ \hline
				latitude & DOUBLE PRECISION & geografska širina \\ \hline
				time & TIMESTAMP & vrijeme kada je lokacija spremljena \\ \hline
				comment & TEXT & komentar za lokaciju \\ \hline

			\end{longtblr}
			
			\subsection{Dijagram baze podataka}
				\begin{figure}[H]
					\includegraphics[scale=0.35]{dijagrami/dijagram_baze_podataka.png}
					\centering
					\caption{Dijagram baze podataka}
					\label{fig:promjene}
				\end{figure}
			\eject
			
			
		\section{Dijagram razreda}
		Dijagram razreda podijeljen je zbog bolje preglednosti na tri dijela: Controllers, Models i DTO. Prikazane su veze koje ostvaruju razredi unutar istog dijela dijagrama, a odnosi između razreda u različitim dijelovima mogu se zaključiti iz tipova atributa. Metode korištene u Controller razredima vraćaju \textit{ResponseEntity}, koji predstavlja HTTP odgovor, ili samo kod odgovora HTTP-a. U svom radu koriste objekte za prijenos podataka (DTO) i ostvaruju komunikaciju s klijentskom stranom.
			
			\begin{figure}[H]
				\includegraphics[scale=0.5]{dijagrami/Controllers.png} 
				\centering
				\caption{Dijagram razreda - dio Controllers}
				\label{fig:promjene}
			\end{figure}
			
			\begin{figure}[H]
				\includegraphics[scale=0.5]{dijagrami/DTO.png} 
				\centering
				\caption{Dijagram razreda - dio DTO}
				\label{fig:promjene}
			\end{figure}
			
			\begin{figure}[H]
				\includegraphics[scale=0.4]{dijagrami/Model.png} 
				\centering
				\caption{Dijagram razreda - dio Models}
				\label{fig:promjene}
			\end{figure}
			
			
			\eject
		
		\section{Dijagram stanja}
			
			
			Dijagrami stanja prikazuju kako sustav prelazi iz jednog stanja u drugo kao odgovor na događaje. Prijavljenom korisniku, u ovom slučaju istraživaču, prikazuje se početna stranica s nekoliko opcija: pregled vlastitih akcija, stvaranje nove akcije i pregled osobnih podataka. Nakon odabira opcije stvaranja nove akcije, korisniku se nudi mogućnost slanja zahtjeva za tragačima za pripadajuću akciju. U slučaju da nema slobodnih tragača, sustav obavijesti korisnika i čeka na idući zahtjev. Ako se istraživaču dodijele tragači, može im dodijeliti zadatke. Zatim ima opciju unosa komentara na zadatak. Odabirom opcije ,,Moje akcije“ , istraživač može vidjeti sve svoje akcije i pregledati njihove članove te završiti neku od akcija.  Također, istraživač može odabrati prikaz karte i urediti je po želji. Nakon toga može ostaviti komentar na karti.
			
			\begin{figure}[H]
				\includegraphics[scale=0.3]{dijagrami/DijagramStanja.png} 
				\centering
				\caption{Dijagram stanja}
				\label{fig:promjene}
			\end{figure}
			
			
			\eject 
		
		\section{Dijagram aktivnosti}
			
			Dijagrami aktivnosti upotrebljavaju se za modeliranje dinamičkog ponašanja sustava. Izvođenje aktivnosti prikazano je kroz niz akcija koje čine upravljačke tokove i tokove objekata. Prikazan je proces stvaranja nove akcije. Aktivnost započinje prijavom istraživača u sustav. Nakon uspješne prijave, istraživaču se prikazuje početna stranica. Odabirom opcije za stvaranje nove akcije aplikacija prikazuje formu za unos podataka o akciji. Istraživač unosi podatke o akciji, a sustav ih pohranjuje u bazu podataka. Zatim istraživač šalje zahtjev za tragačima, a aplikacija prikazuje formu za unos detalja o zahtjevu. Aplikacija prosljeđuje zahtjev voditelju postaje koji dodjeljuje tragače akciji (pretpostavlja se da ima dostupnih tragača). Popis tragača se pohranjuje u bazu i prikazuje u aplikaciji. Na zahtjev za dodjeljivanje zadatka nekom od tragača prikazuje se forma za unos zadatka. Zadatak se sprema u bazu podataka, a istraživač može unijeti i komentar o zadatku. Nakon završetka unosa svih podataka o akciji aplikacija prikazuje poruku da su svi podaci spremljeni i aktivnost završava.
			
			\begin{figure}[H]
				\includegraphics[scale=0.4]{dijagrami/DijagramAkt.png} 
				\centering
				\caption{Dijagram aktivnosti}
				\label{fig:promjene}
			\end{figure}
			
			\eject
		\section{Dijagram komponenti}
		
			\textbf{\textit{dio 2. revizije}}\\
		
			 \textit{Potrebno je priložiti dijagram komponenti s pripadajućim opisom. Dijagram komponenti treba prikazivati strukturu cijele aplikacije.}